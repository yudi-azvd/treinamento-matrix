Nesse treinamento vamos aprender algumas ferramentas básicas de web pra imitar a chuva de código do filme Matrix. O resultado pode ser visualizado \href{https://yudi-azvd.github.io/matrix/}{aqui}. 
E esse trabalho foi inspirado pelo \href{https://www.youtube.com/watch?v=S1TQCi9axzg}{tutorial} da Emily Xie.

Para desenvolver esse projeto vamos utilizar HTML, CSS e JavaScript. Esse projeto poderia ser desenvolvido em C, Java ou Python, usando bibliotecas de animação ou não. Entretanto, o objetivo principal desse treinamento é aprender alguns conceitos e técnicas de programação pra melhorar a qualidade e legibilidade de código. Conceitos e técnicas como \textbf{abstração}, \textbf{divisão de problemas em subproblemas}, e \textbf{convenções} para nomeação de variáveis e entre outros.

Antes de prosseguir certifique-se de ter algum editor de texto voltado para programação instalado em seu computador. Recomendo fortemente o 
\href{https://code.visualstudio.com/}{Visual Studio Code}(VS code). Outros como 
\href{https://www.sublimetext.com/}{Sublime Text} ou 
\href{https://atom.io/}{Atom} também são muito bons.
