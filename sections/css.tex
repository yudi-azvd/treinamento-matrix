Ou \emph{Cascading Style Sheets} (CSS) \textbf{é o que dá estilo pra
página e, se bem utilizado, beleza também. }Vamos rir - didaticamente e
para fins de visualização - de algumas
\href{https://www.elegantthemes.com/blog/resources/bad-web-design-a-look-at-the-most-hilariously-terrible-websites-from-around-the-web}{páginas
web que se encontram pela internet} que poderiam muito bem ser dos anos
2000.

Continuando nossa analogia do prédio, o CSS cuidaria do acabamento das
paredes (gesso, tinta, papeis de parede e quadros), placas de
orientação, corrimãos, tapetes, janelas e talvez cadeiras e mesas.

É baseado em seletores e regras. \textbf{Seletores} escolhem os
elementos de interesse e as \textbf{regras}
são as estilizações em si:
fundo preto, itálico, fonte courier por exemplo. Com CSS podemos mudar
cor, fonte, tamanho, transparência, fundo, margem, padding
(preenchimento) e até o block level dos elementos.
